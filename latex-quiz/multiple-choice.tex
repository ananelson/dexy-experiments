% http://frigg.physastro.mnsu.edu/~eskridge/astr101/sample.html
% 
% Astronomy 101 Sample Test Questions
% 
% The following are 98 questions taken from my question bank. In all cases the
% correct answer is the first answer given (the one marked *). Note that in an
% actual exam, the order of the answers is scrambled.

  1) The light-year is a measure of
 
 *)distance
 1)time                                       
 2)speed
 3)weight                         
 4)brightness           
 
 2) About how long does it take light from the Sun to reach the Earth?
 
 *)About 8 minutes.
 1)About 8 seconds.
 2)No time at all.
 3)About 4 years.
 4)About a week.
 
 3) The light-second is a measure of
 
 *)distance
 1)time                                       
 2)speed
 3)weight                         
 4)brightness           
 
 4) The zenith is
 
 *)the point on the sky directly overhead.
 1)the point on the sky directly overhead from the North Pole.
 2)the point on the sky on the horizon to the due south.
 3)the point on the sky directly overhead from the Equator.
 4)the point on the sky the Sun occupies.
 
 5) Constellations are 
 
 *)patterns of stars in the sky.
 1)clusters of stars very near each other in space.
 2)groupings of planets in the sky.
 3)close associations of stars and visible planets.
 4)non-existant.
 
 6) A star seen at the zenith from Mankato must be
 
 *)north of the celestial equator
 1)south of the celestial equator
 2)east of the celestial equator
 3)west of the celestial equator
 4)exactly on the celestial equator
 
 7) The apparent magnitude of a star is a measure of its
 
 *)brightness as seen from earth
 1)intrinsic radiance
 2)rate of energy output at all wavelengths
 3)brightness if it was at a distance of ten parsecs
 4)angular diameter as seen from the earth
 
 8) The absolute magnitude of a star is a measure of its
 
 *)brightness if it was at a distance of ten parsecs
 1)brightness as seen from earth
 2)angular size
 3)rate of energy output at all wavelengths
 4)angular diameter as seen from the earth
 
 9) If the inclination of the Earth's rotational axis increased
 
 *)winters would be cooler
 1)summers would be cooler
 2)winters would be warmer
 3)summers would be the same
 4)winters would be the same
 
10) The seasons we experience on the Earth are caused by
 
 *)the tilt of the Earth's axis with respect to the ecliptic pole
 1)the change in the Earth's distance from the sun during the year
 2)the change in the luminosity of the Sun during the year 
 3)the change in the Moon's distance from the Earth during the year
 4)the precession of the Earth's pole
 
11) The Solar day is
 
 *)about 4 minutes longer than the Sidereal day.
 1)about 4 minutes shorter than the Sidereal day.
 2)equal to the Sidereal day.
 3)just over 27 Sidereal days long.
 4)just over 29 Sidereal days long.
 
12) The precession of the Earth
 
 *)causes the position of the Celestial Poles to change very slowly.
 1)causes rapid, large changes in the position of the Celestial Poles.
 2)was only discovered in the last century.
 3)is making the Earth spiral into the Sun.
 4)is causing the rotation rate of the Earth to decrease.
 
13) If the moon was new a week ago, what would its phase be now?
 
 *)first quarter
 1)full
 2)third quarter
 3)waxing gibbous
 4)waning crescent
 
14) The sidereal month is
 
 *)about two days shorter than the synodic month.
 1)about two days longer than the synodic month.
 2)equal to the Solar day.
 3)just under 24 hours long.
 4)just over 29 Solar days long.
 
15) If the Moon rises at sunset, what is its phase?
 
 *)Full
 1)1st quarter
 2)3rd quarter
 3)New 
 4)Waning gibbous
 
16) If two objects of different masses are dropped (ignoring air friction)
 
 *)they will fall and accelerate at the same rate
 1)the more massive object will fall faster
 2)the less massive object will fall faster
 3)they will fall at the same constant rate
 4)the more massive object will fall at a constant speed
 
17) An accelerating body must
 
 *)have a changing velocity.
 1)always have a positive velocity.
 2)be changing direction.
 3)be changing speed.
 4)be falling.
 
18) Spring tides occur near which lunar phases?
 
 *)new and full
 1)first quarter and third quarter
 2)full and last quarter
 3)full and third quarter
 4)new and first quarter
 
19) Newton's Force Law relates
 
 *)1 and 2 are correct.
 1)the acceleration an object experiences with its mass.
 2)the accelaration an object experiences with the force applied to it.
 3)the mass of an object to its composition.
 4)the acceleration an object experiences with its composition.
 
20) The length of the Moon's rotation period
 
 *)is equal to its orbit period due to the tidal effect of the Earth.
 1)is gradually decreasing due to tidal drag from the Earth.
 2)has not changed since the Moon formed.
 3)is about 24 hours long.
 4)will eventually be equal to a Solar year.
 
21) While the Moon and the Sun both exert tidal forces on the Earth
 
 *)the Moon's are larger because it is so much closer to Earth than
   the Sun is.
 1)the Sun's are larger because it is so much more massive than the
   Moon is.
 2)the Sun's are larger because it is so much closer to Earth than the
   Moon is.
 3)the Moon's are larger because it is so much more massive than the
   Sun is.
 4)they are equal and opposite to one another, so cancel each other out.
 
22) The acceleration due to gravity on the surface of the Moon
 
 *)is less than it is on the Earth because the Moon's mass is much
   smaller than the Earth's mass.
 1)is more than it is on the Earth because the Moon's mass is much
   smaller than the Earth's mass.
 2)is equal to what it is on the surface of the Earth.
 3)is less than it is on the Earth because the Moon's mass is much
   larger than the Earth's mass.
 4)is more than it is on the Earth because the Moon's mass is much
   larger than the Earth's mass.
 
23) The Dynamic Range of a detector is a measure of 
 
 *)the range between the faintest and brightest sources it can measure
 1)the range of photon energies it can measure
 2)the range of motion it can withstand without breaking
 3)the range in quantum efficiency it provides
 4)the range of radial velocities it can measure
 
24) The reason why new mirror designs were needed to build 8-10m 
    telescopes is that an 8m mirror of the traditional design
 
 *)is too heavy to support
 1)is too fragile
 2)is too floppy
 3)can't be properly polished
 4)is too light to control
 
25) Observations of astronomical objects at wavelengths across the
    electromagnetic spectrum are so useful because
 
 *)different states of matter radiate in very different wavelengths.
 1)our eyes see such a large fraction of the electromagnetic spectrum.
 2)the only source of electromagnetic radiation is stars.
 3)very hot matter does not radiate.
 4)visible light does not tell us anything about stars.
 
26) The more strongly curved a mirror is
 
 *)the shorter its focal length.
 1)the longer its focal length.
 2)the bluer the reflected light.
 3)the redder the reflected light.
 4)the better it makes you look.
 
27) All large telescopes built in the last hundred years have been
    reflectors.  This is because
 
 *)1 2 and 3 are all true.
 1)refractors suffer from chromatic aberration but reflectors do not.
 2)it is more challenging to make a good lens than a good mirror.
 3)the mass of the lens is at the top of the telescope, but the mass of
   the mirror is at the bottom.
 4)reflectors suffer from chromatic aberration refractors do not.
 
28) The aperture of a telescope refers to 
 
 *)the diameter of its primary mirror or lens.
 1)the length of the telescope tube.
 2)the diameter of its secondary mirror.
 3)the size of the eye-piece lens.
 4)the curvature of the primary mirror or lens.
 
29) Light from the optical part of the spectrum 
 
 *)is mostly emitted from the surfaces of stars.
 1)is mostly emitted from small grains of solid matter.
 2)is mostly emitted from very hot gas in interstellar space.
 3)tells us about the nature of very cold matter.
 4)tells us about the nature of very hot, diffuse matter.
 
31) If one person shines a flashlight (F1) at you from 10m away, and a
    second person shines an identical flashlight (F2) at you from 20m
    away, what is the relative amount of light you measure from the two
    flashlights?
 
 *)F1 appears 4 times brighter than F2
 1)F1 appears 2 times brighter than F2
 2)F1 appears 10 times brighter than F2
 3)F1 and F2 appear equally bright
 4)F2 appears 4 times brighter than F1
 
32) The frequency of a light wave is inversely proportional to its
 
 *)wavelength
 1)energy
 2)speed
 3)velocity
 4)amplitude
 
33) The radial velocity of a star can be determined from its
 
 *)spectral line Doppler shift                                       
 1)distance and luminosity                                           
 2)spectral type                                                     
 3)proper motion                                                     
 4)right ascension and declination
 
34) If we compare two stars of different temperature, the hotter star
 
 *)emits more energy from each unit area of surface
 1)will have a brighter absolute magnitude
 2)will have a brighter apparent magnitude
 3)will appear redder
 4)will be larger
 
35) An absorption spectrum is caused by 
 
 *)atoms in a diffuse gas absorbing photons of particular wavelengths from
   and underlying continuum source.
 1)a hot diffuse gas
 2)completely blocking the light from a continuum source
 3)emission from a hot opaque object
 4)atoms in a diffuse gas emitting photons of particular wavelengths
 
36) Neils Bohr's model of the hydrogen atom explains the observed lines
    in the hydrogen spectrum as a consequence of 
 
 *)the existence of a set of allowed orbital energies for the electron
   in a hydrogen atom.
 1)the existence of a set of allowed orbital energies for the proton in
   a hydrogen atom.
 2)the existence of a set of allowed orbital energies for the neutron in
   a hydrogen atom.
 3)the lack of any restriction on the allowed orbital energies for the
   electron in a hydrogen atom.
 4)the annhialation of electrons by antimatter.
 
37) Two atoms of different elements must have
 
 *)different numbers of protons.
 1)the same number of protons.
 2)the same number of neutrons.
 3)different numbers of neutrons.
 4)the same number of protons and neutrons.
 
38) The electrostatic force between an neutron and a proton is
 
 *)zero, because the neutron has no charge.  
 1)repulsive because they have opposite charges
 2)attractive because they have equal charges
 3)repulsive because they have equal charges
 4)zero because they have opposite charges
 
39) To find the total mass of a visual binary, we need to first find its
 
 *)distance from earth and the apparent orbit
 1)Doppler shift and proper motion
 2)true orbit and proper motion
 3)apparent magnitude and distance from earth
 4)space velocity and apparent orbit
 
40) The spectral sequence from M to O is a sequence of
 
 *)increasing temperature
 1)decreasing age
 2)decreasing mass
 3)decreasing temperature
 4)decreasing ionization
 
41) The type of spectrum emitted by most stars is
 
 *)continuous with absorption lines
 1)pure continuous
 2)continuous with emission lines
 3)pure emission
 4)pure absorption
 
42) We know the stars called Red Giants are much larger than the Sun
    because
 
 *)they are cool and luminous, so the Stefan-Boltmann law argues they
   must be large.
 1)all stars are larger than the Sun.
 2)they are hot and luminous, so the Stefan-Boltmann law argues they
   must be large.
 3)they are cool and faint, so the Stefan-Boltmann law argues they
   must be large.
 4)they are hot and faint, so the Stefan-Boltmann law argues they
   must be large.
 
43) Stars transport energy from their cores to their surfaces
 
 *)1 and 2 are correct.
 1)by radiative diffusion.
 2)by convection.
 3)by conduction.
 4)1 and 3 are correct.
 
44) In following the Main Sequence on the H-R diagram in the direction of
    increasing temperature, one is also following a sequence of
 
 *)increasing mass
 1)decreasing mass
 2)decreasing luminosity
 3)decreasing stellar diameter
 4)increasing age
 
45) The Stefan-Boltzmann relation allows us to draw lines on the HR 
    Diagram that are constant
 
 *)in radius.
 1)in age.
 2)in mass.
 3)in density
 4)in composition
 
46) The interstellar medium consists of
 
 *)All of the other answers are correct
 1)ionized hydrogen gas
 2)neutral hydrogen gas
 3)interstellar dust grains
 4)interstellar molecules
 
47) Why doesn't the sun presently create energy by fusing helium into
    heavier elements?
 
 *)The sun's core isn't hot and dense enough
 1)There's no helium in the sun's core
 2)When helium fuses, no energy is ever released
 3)Helium nuclei cannot fuse to make heavier atoms
 4)Neutrinos inside the sun prevent helium fusion
 
48) Interstellar extinction, or reddening
 
 *)is caused by small interstellar dust particles
 1)is caused by gas atoms and molecules
 2)is due to a Doppler red shift
 3)is a very minor and unimportant phenomenon
 4)is caused by golf-ball-sized particles in space
 
49) Why are protostars so difficult to observe?
 
 *)They emit most of their radiation in the IR.
 1)They do not emit photons.
 2)They emit most of their radiation as x-rays.
 3)Stars are no longer forming in our Galaxy.
 4)They are too small to have high luminosities.
 
50) Reflection nebulae are
 
 *)caused by the scattering of light off dust grains around young stars
 1)due to emission-line radiation from hot gas
 2)generally red because red light scatters more than blue light
 3)only seen around very old stars
 4)clouds of very hot, dense gas
 
51) We can probe the internal structure of molecular clouds by
 
 *)observing lines of many different molecules that tell us about gas
   at a range of temperature and density.
 1)watching cloud cores collapse in real time.
 2)making visible-wavelength observations of collapsing cloud cores.
 3)making x-ray observations of the hot gas.
 4)observing thermal emission from very cold dust.
 
52) The coldest phase of the interstellar medium is
 
 *)molecular gas at 10 K that we detect from CO radio line emission.
 1)molecular gas at 10 K that we detect from hydrogen line emission.
 2)atomic gas at 10 K that we detect from hydrogen line emission.
 3)atomic gas at 100 K that we detect from hydrogen line emission.
 4)atomic gas at 100 K that we detect from CO radio line emission.
 
53) A star cluster is a useful natural `laboratory' for studying stellar
    evolution because the stars in a cluster all
 
 *)have similar age and similar distance.
 1)have about the same mass.
 2)have similar spectral type.
 3)are in binary systems.
 4)are main-sequence stars.
 
54) The elements heavier than hydrogen and helium that we find on earth
    probably were created
 
 *)inside a star that is no longer visible.
 1)in the sun.
 2)when the universe was created.
 3)inside the earth.
 4)None of the other answers is correct.
 
55) Horizontal Branch stars are
 
 *)roughly Solar mass stars with Helium-burning cores
 1)roughly Solar mass stars with Hydrogen-burning cores
 2)roughly Solar mass stars with inert Helium cores
 3)roughly Solar mass stars with inert Carbon cores
 4)roughly Solar mass stars with inert Hydrogen cores
 
56) Red Giant Branch stars are
 
 *)stars with inert Helium cores and Hydrogen-burning shells.
 1)stars with inert Carbon/Oxygen cores and Hydrogen and Helium-burning 
   shells.
 2)stars with Helium-burning cores and Hydrogen-burning shells.
 3)stars with Carbon-burning cores and Hydrogen and Helium burning shells
 4)stars with Hydrogen-burning cores and Helium-burning shells.
 
57) The triple-alpha process is
 
 *)the process that evolved stars use to turn Helium into Carbon.
 1)the process that the Sun uses to turn Hydrogen into Helium.
 2)the process that evolved stars use to turn Hydrogen into Helium.
 3)the process that evolved stars use to turn Carbon into Hydrogen.
 4)the process that the Sun uses to turn Helium into Hydrogen.
 
58) Mass flowing from a star to a compact object (White Dwarf, Neutron
    Star, or Black Hole)
 
 *)1 and 2 are correct.
 1)first forms an accretion disk around the compact object.
 2)is heated sufficiently to radiate in the UV or x-ray.
 3)falls directly onto the compact object.
 4)cannot ever be accreted by the compact object.
 
59) Type Ia supernova explosions result from
 
 *)mass transfer onto white dwarfs.
 1)iron core collapse.
 2)He-burning.
 3)mass loss.
 4)star-star collisions.
 
60) White dwarfs are the remnants of low mass stars supported by
  
 *)degenerate electron pressure
 1)degenerate neutron pressure
 2)rapid rotation
 3)magnetic field pressure
 4)substantial thermonuclear fusion
 
61) An important difference between novae and supernovae is that novae
 
 *)return to their previous brightness after a few years.
 1)completely destroy their parent stars.
 2)cause neutron stars to form.
 3)rarely occur in close binary systems.
 4)expel more heavy atoms into space than supernovae do.
 
62) The Schwarzschild Radius of a black hole depends only on the
    ________ of the collapsing stellar core.
 
 *)mass
 1)density
 2)temperature
 3)pressure
 4)composition 
 
63) Einstein concluded that mass
 
 *)causes space to curve
 1)increases the speed of light
 2)has no effect on light 
 3)decreases as you go faster
 4)has no relationship with energy
 
64) Einstein concluded that an observer isolated in a small box, but
    feeling his normal weight, could not tell if he were
 
 *)on the surface of the Earth or being accelerated by a rocket in space
 1)on the Earth or the moon
 2)on a neutron star or a white dwarf
 3)traveleing through time
 4)having his clock sped up by a nearby mass
 
65) Gravitational lenses occur because
 
 *)gravity can bend light
 1)the speed of light depends on wavelength
 2)the interstellar medium refracts light
 3)glass lenses are very massive
 4)objects in the sky tend to be clustered
 
66) The final collapse of the cores of evolved high-mass stars 
 
 *)1 and 2 are correct.
 1)is halted by the onset of Neutron Degeneracy Pressure.
 2)occurs when the core reaches nuclear density.
 3)is halted by the onset of Electron Degeneracy Pressure.
 4)is halted by the onset of thermonuclear fusion.
 
67) What evidence do we have that Black Holes exist?
 
 *)We see binary systems in which one member produces no optical emission,
   but appears to have too large a mass to be a neutron star.
 1)We see isolated black spots in space.
 2)We see binary systems in which one member produces no optical emission,
   but appears to have too small a mass to be a neutron star.
 3)Nearby moving Black Holes are seen to produce gravitational distortions
   in the images of background stars.
 4)We have no evidence to support the existence of Black Holes.
 
68) Our Milky Way galaxy consists of
 
 *)a disk, central bulge, and halo.
 1)a disk or plane only.
 2)a spherical distribution of stars centered on the sun.
 3)a disk containing gas, dust, and globular clusters.
 4)a sphere of stars centered on Sagittarius.
 
69) Which of the following are not likely to be found in the spiral arms
    of our Galaxy?
 
 *)globular clusters
 1)H II regions
 2)young star clusters
 3)emission nebulae
 4)clouds of gas and dust
 
70) The concept of stellar populations is used to show an association
    between
 
 *)location in the Galaxy, chemical composition, and age
 1)distance from the sun, orbital speed, and age
 2)distance from the Galactic Center and orbital speed
 3)distance from the sun, chemical composition, and age
 4)temperature, mass, and distance from the Galactic Center
 
71) The globular clusters in our Galaxy
 
 *)are spherically distributed about the galactic center.
 1)are concentrated in the galactic nucleus.
 2)are present only in our galaxy.
 3)are concentrated in spiral arms.
 4)are concentrated near the edge of the galaxy.
 
72) About how many trips has the Sun made around the Galaxy since it
    formed?
 
 *)two dozen
 1)one tenth
 2)one thousand
 3)one million
 4)one billion
 
73) The rotation curve of the Galaxy is flat at large radii.  This
    means that
 
 *)the mass of the Galaxy is increasing with radius out to the most
   distant points we can measure.
 1)the mass of the Galaxy is entirely accounted for by the stars.
 2)most of the mass of the Galaxy is in the form of hydrogen gas.
 3)the mass of the Galaxy is nearly all at its center.
 4)the Sun is escaping from the Galaxy.
 
74) Density Wave theory 
 
 *)is an explanation of spiral structure due to waves of compression.
 1)is believed to explain spiral structure in all spiral galaxies.
 2)explains the association of spiral patterns with the distribution
   of the oldest stars in the Galaxy.
 3)is not considered important for the structure of the Galactic disk.
 4)explains the structure of the Galactic halo.
 
75) The Galactic Bulge
 
 *)shows a mix of Population I, Population II, and old, metal-rich
   stars.
 1)is a pure Population I system.
 2)is a pure Population II system.
 3)shows a mix of Population I, Population II and young, metal-poor
   stars.
 4)is impossible to observe.
 
76) Hubble discovered that spiral nebulae were outside our Galaxy by
    observing
 
 *)Cepheid variables.
 1)supernovae.
 2)novae.
 3)RR Lyrae stars.
 4)the rotation of the nebulae.
 
77) A large cluster of galaxies is likely to have
 
 *)hot gas between the galaxies, emitting X-rays
 1)cold gas between the galaxies, emitting X-rays
 2)cold gas between the galaxies, absorbing X-rays
 3)hot gas between the galaxies, absorbing radio waves
 4)dark matter between the galaxies absorbing visible light
 
78) The key to discovering that spiral objects seen in the sky were really
    separate galaxies was
 
 *)measuring their distances
 1)proper motion of the objects
 2)radial velocity showing rotation of the objects
 3)measuring the parallax of the objects
 4)measuring their sizes
 
79) The greater the distance of a galaxy away from the earth
 
 *)the greater the observed red shift.
 1)the older it is.
 2)the greater the observed blue shift.
 3)the more massive it is.
 4)None of the choices given here is correct.
 
80) Dark matter is known to exist because of
 
 *)its gravitational influence on visible matter
 1)obscuration of distant galaxies
 2)current star formation rates in the Milky Way
 3)directly observable x-ray emission
 4)none of the other answers is correct
 
81) Spiral and elliptical galaxies differ in their
 
 *)All of the other answers are correct
 1)morphology
 2)gas and dust content
 3)stellar orbits
 4)internal ranges of stellar ages
 
82) Some very nearby galaxies have only recently been detected because
 
 *)1 and 2 are true.
 1)they have very low surface brightness.
 2)they lie behind the plane of the Milky Way.
 3)no one started looking until recently.
 4)they emit no optical photons.
 
83) A major problem with current observations of high-redshift galaxies is
    is that
 
 *)all the other answers are correct.
 1)galaxy images are only a few pixels across with our best instruments.
 2)distant galaxies are very faint, and so only the brightest parts are
   visible.
 3)optical images of high-redshift galaxies are showing us what they
   look like in the ultraviolet part of the spectrum.
 4)near infrared images, that would show us the visible-wavelength
   appearence of high-redshift galaxies are not sensitive enough.
 
84) Evidence that quasars are far away is provided by
 
 *)their large redshifts
 1)their strong X-ray emission
 2)the jets seen with radio telescopes
 3)their rapidly changing brightness
 4)the strength of their emission lines
 
85) It appears that one reason different active nuclei of galaxies have
    such a variety of spectra results from
 
 *)the angle at which we view each nucleus
 1)the amount of dust in our Galaxy blocking the view
 2)the different masses of the galaxies
 3)whether or not the galaxy has a close companion
 4)how many neutron stars they contain
 
86) What is the evidence that much of the light from a quasar comes
    from a remarkably small volume?
 
 *)Quasars can vary in brightness rapidly.
 1)Quasars can vary in position rapidly.
 2)Quasar redshifts can change rapidly.
 3)Quasars look small through a telescope.
 4)Quasars have a very small proper motion.
 
87) Seyfert galaxies are
 
 *)spiral galaxies with very bright nuclei
 1)exploding elliptical galaxies
 2)spiral galaxies which lack interstellar gas and dust
 3)found only in our Local Group of galaxies
 4)the most distant objects ever seen in the universe
 
88) If quasar redshifts are interpreted with the Hubble Law for normal
    galaxies, this implies that quasars
 
 *)must be very distant and therefore very luminous
 1)must be very nearby and faint
 2)must have very large radii
 3)show no peculiarity except for their large distances
 4)do not exist
 
89) Why do we conclude that the broad emission lines seen in the spectra
    of Seyfert galaxies are caused by orbital velocities?
 
 *)If the line-widths were thermal, the gas would be much too hot to
   radiate optical emission lines.
 1)If the line-widths were thermal, the gas would be much too cold to
   radiate optical emission lines.
 2)The required orbital velocities are about the same as the orbital
   velocities of stars in the Galactic disk.
 3)We can take images that resolve the individual orbiting gas clouds.
 4)We can measure the proper motions of the individual orbiting gas clouds.
 
90) In the 1950s, one of the really surprizing things about Cyg A was
 
 *)that one of the brightest radio sources in the sky could be so far away.
 1)that one of the brightest radio sources in the sky could be so nearby.
 2)that its optical luminosity could be so large.
 3)that one of the brightest optical sources in the sky could be so far 
   away.
 4)that it produces no radio emission.
 
91) BL Lac objects are
 
 *)active galaxies with featurless continuum spectra.
 1)a peculiar class of variable star.
 2)characterized by strong emission lines in their spectra.
 3)active galaxies that have radio jets perpendicular to our line of sight.
 4)not detected in visible light.
 
92) If the universe is infinite and Newtonian, the sky should be uniformly
    bright.  Clearly it is not.  The preceding is a statement of
 
 *)Olbers's paradox
 1)the perfect cosmological principle
 2)Einstein's postulate
 3)the cosmological principle
 4)Hubble's paradox
 
93) The distance to the most distant thing in the observable universe is
    set by the
 
 *)age of the Universe
 1)mass of the Universe
 2)effects of gravity
 3)most distant Cepheid variables we can see
 4)blockage of light by interstellar dust
 
94) The Hubble constant measures the
 
 *)current expansion rate of the universe
 1)density of the universe
 2)curvature of space in the universe
 3)temperature of the universe
 4)distance to galaxies in the Local Group
 
95) One major success of the Big Bang Theory for the origin of the
    universe is that it predicted the
 
 *)cosmic background radiation
 1)existence of galaxies
 2)amount of iron atoms in the universe
 3)spin of spiral galaxies
 4)blueshifts of quasars
 
96) Our current understanding is that the night sky is dark because
 
 *)the Universe had a beginning
 1)there are large quantities of dust
 2)the light from distant stars is converted into neutrinos
 3)No one knows
 4)distant stars are gravitationally lensed
 
97) To study the Universe as it was long ago, we need to
 
 *)observe very distant objects
 1)detect neutrinos
 2)only observe in radio waves
 3)determine the composition of dark matter
 4)travel to the edge of the Universe
 
98) One succesful test of the Big Bang model for the early Universe is
    that it predicts
 
 *)the observed abundances of helium and other light isotopes produced 
   during the epoch of primordial nucleosynthesis
 1)that it predicts that all the observed helium is produced in stars
 2)the production of carbon and oxygen during the epoch of primordial
   nucleosynthesis
 3)the existence of quasars
 4)the cosmic x-ray background
 

